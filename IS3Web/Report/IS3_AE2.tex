\documentclass[a4paper]{article}
\usepackage{graphicx}
\usepackage[margin=1in]{geometry}
\usepackage{caption}
\usepackage{hyperref}


\title{Interactive Systems 3\\ \vspace{4mm} 
Group report for Assessed Exercise Part 2}

\author{\bf Adam G.(1107425), Belev A.(1103816), Ciesielczuk R. (1104712),\\ \bf Georgiev M.(1107308), Gliebus D.(1101651),\\ \bf Fairclough D. (1006103), Kotev R.(1103815)}

\date{\today}

\begin{document}
\maketitle
\newpage



\section{Introduction}
This is a one page introduction to the system developed for the Interactive Systems Assessed Exercise. 
\subsection{Idea}
The application's main idea is to visualize significant features in the data set and provide the user with the ability to explore relations in the data set.
\subsection{Technologies used} The main technologies used were JQuery and Google Charts.
\subsection{How to run it} Extract the zip archive in a folder and the...



\section{Software Design Requirements}
We will discuss each requirement in the following way: first we will describe the motivation behind the requirement (which is extended from the report of the first part of the assessed exercise and then we will describe the requirement in a software engineering way.\\
The first part of this section will describe requirements described in sections 4 and 5 of the report of the first part of the assessed exercise. The second part will describe requirements that were thought of during the implementation of the system. The third part will discuss possible requirements for further development of the system.
\begin{itemize}
\subsection{Initial requirements}
\item{\textbf{Chart switching panel}\\Motivation: relations between different attributes are best represented with different charts. The user must be able to quickly switch between different charts and to preserve its attribute selection.\\The user should be able to switch between the following types of charts: Line Chart, Bar Chart, Scatter Chart and Geo Map. If the user has selected a specific set of attributes to visualize, switching between charts should preserve this user choice (i.e. if a user is comparing Gold and Silver medals for countries in Africa in a Line chart, when he switches to Bar chart, only gold and silver medals will be plotted on the Bar chart).}
\item{\textbf{Attribute selection panel}\\Motivation: the user must be able to explore different sets of attributes\\The user should be able to add attributes to be plotted, to remove attributes of those that are plotted, to deselect all and to select all attributes. If no attribute is selected, then a message should be displayed that prompt the user to select an attribute. There should be a column filer.}
\item{\textbf{Country selection panel}\\Motivation: the user must be able to explore different sets of counties\\The user should be able to add countries to be plotted, to remove countries of those that are plotted, to deselect all and to select all countries. If no countries is selected, then a message should be displayed that prompt the user to select an attribute. There should be a country filter.}
\item{\textbf{Multiple Line chart}\\Motivation: The user want to explore all the countries. A significant advantage of a line chart is that it visualizes well a large number of data points ( 180 countries in our case). The user might want to view more than one attribute of a country and that is why we chose to make a multiple line chart. The countries are plotted on the x-axis.\\The user must be able to view attributes plotted as a bar chart}
\item{\textbf{Bar chart}\\The user want to explore a small set of countries. A significant advantage of a bar chart is that it visualizes well a small number of data points. The user might want to view more than one attribute of a country and that is why we chose to make a multiple bar chart. The countries are plotted on the x-axis.\\The user must be able to view attributes plotted as a bar chart}
\item{\textbf{Scatter chart}\\Motivation: Scatter charts are good when exploring for a correlation of attributes that are comparable (we do not want to plot the countries on the x-axis, i.e. we want to plot gold medals on the x-axis and silver medals on the y-axis.\\The user must be able to view attributes plotted as a scatter chart}

\subsection{On the go requirements}
\item{\textbf{Geo Map}\\Motivation: Geo Maps are good when considering geographical factors\\Include a Geo Map chart option available in the chart selection panel.}
\item{\textbf{Data Normalization}\\Motivation: Normalization is useful when comparing attributes whose units different by great magnitude.\\The user should be able to normalize the data.}
\item{\textbf{Chart zoom}\\Motivation: The user might want to zoom a particular sector of the chart in order to see the attributes better.\\The user might be able to zoom a chart.}

\subsection{Further development requirements}
\item{\textbf{File upload}\\Motivation: explore different data sets\\The user should be able to upload different data sets.}

\end{itemize}



\section{Implementation}



\section{Analytic Findings}
Here we will write some correlations and other results we obtained when analyzing the data.
\end{document}